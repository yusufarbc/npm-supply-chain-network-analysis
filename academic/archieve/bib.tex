% APA-style bibliography derived from academic/literatüre.md
\begin{thebibliography}{30}

\bibitem{lit1} Wyss, E. (2025). A new frontier for software security: Diving deep into npm.

\bibitem{lit2} Jaisri, P., Reid, B., \& Kula, R. G. (2024). A preliminary study on self-contained libraries in the NPM ecosystem.

\bibitem{lit3} Yu, S. (2024). Accurate and efficient SBOM generation for software supply chain security.

\bibitem{lit4} Ohm, M., Plate, H., Sykosch, A., \& Meier, M. (2020). Backstabber's knife collection: A review of open source software supply chain attacks.

\bibitem{lit5} Rahman, I., Zahan, N., Magill, S., Enck, W., \& Williams, L. (2024). Characterizing dependency update practice of NPM, PyPI and Cargo packages 

\bibitem{lit6} Hastings, T. G. (2024). Combating source poisoning and next-generation software supply chain attacks using education, tools, and techniques 

\bibitem{lit7} Wang, M., Wu, P., \& Luo, Q. (2023). Construction of software supply chain threat portrait based on chain perspective 

\bibitem{lit8} Liu, C., Chen, S., et al. (2022). Demystifying vulnerability propagation via dependency trees in npm. In Proceedings of the 44th International Conference on Software Engineering (ICSE 2022).

\bibitem{lit9} Ahlstrom, H. E. (2025). Dependency analysis for software licensing and security 

\bibitem{lit10} Javan Jafari, A., Costa, D. E., Abdellatif, A., \& Shihab, E. (2023). Dependency practices for vulnerability mitigation 

\bibitem{lit11} Correia, M. L. P. (2022). Detection of software supply chain attacks in code repositories 

\bibitem{lit12} Kang Yip, D. Y. (2022). Empirical study on dependency-based attacks in Node.js.

\bibitem{lit13} Torres-Arias, S. (2020). In-toto: Practical software supply chain security.

\bibitem{lit14} Zhang, J., Huang, K., Chen, B., Wang, C., Tian, Z., \& Peng, X. (2023). Malicious package detection in NPM and PyPI using a single model of malicious behavior sequence.

\bibitem{lit15} Halder, S., Bewong, M., Mahboubi, A., Jiang, Y., Islam, R., Islam, Z., Ip, R., Ahmed, E., Ramachandran, G., \& Babar, A. (2024). Malicious package detection using metadata information.

\bibitem{lit16} Hafner, A., Mur, A., \& Bernard, J. (2021). Node package manager's dependency network robustness.

\bibitem{lit17} Ladisa, P., Ponta, S. E., Ronzoni, N., Martinez, M., \& Barais, O. (2023). On the feasibility of cross-language detection of malicious packages in npm and PyPI. 

\bibitem{lit18} Zerouali, A., Mens, T., Decan, A., \& De Roover, C. (2022). On the impact of security vulnerabilities in the npm and RubyGems dependency networks.

\bibitem{lit19} Sejfia, A., \& Schafer, M. (2022). Practical automated detection of malicious npm packages (Amalfi).

\bibitem{lit20} Zimmermann, M., Staicu, C.-A., Tenny, C., \& Pradel, M. (2019). Small world with high risks: Security threats in npm.

\bibitem{lit21} Schorlemmer, T. R. (2024). Software supply chain security: Attacks, defenses, and signing adoption. 

\bibitem{lit22} Cogo, F. R. (2020). Studying dependency maintenance practices through mining NPM.

\bibitem{lit23} Ohm, M., Kempf, L., Boes, F., \& Meier, M. (2021). Supporting detection via unsupervised signature generation (ACME). 

\bibitem{lit24} Ladisa, P., Sahin, M., Ponta, S. E., Rosa, M., Martinez, M., \& Barais, O. (2023). The hitchhiker's guide to malicious third-party dependencies.

\bibitem{lit25} Oldnall, E.-R. (2017). The web of dependencies: A complex network analysis of the NPM.

\bibitem{lit26} Imtiaz, N. (2023). Toward secure use of open source dependencies.

\bibitem{lit27} Vaidya, S. (2022). Towards ensuring integrity and authenticity of software repositories.

\bibitem{lit28} Duan, R., Alrawi, O., Kasturi, R. P., Elder, R., Saltaformaggio, B., \& Lee, W. (2020). Towards measuring supply chain attacks on package managers.

\bibitem{lit29} Zheng, X., Chen, W., Wang, S., Zhao, Y., Gao, P., Zhang, Y., Wang, K., \& Wang, H. (2024). Towards robust detection of OSS supply chain poisoning (OSCAR).

\bibitem{lit30} Shcherbakov, M., Moosbrugger, P., \& Balliu, M. (2021). Unveiling the invisible: Prototype pollution gadgets via dynamic taint.

\end{thebibliography}
