% Bu doküman IEEE Türkiye tarafından düzenlenen TUAC için oluşturulmuş taslaktır. Yalnızca içeriklerin değiştirilmesi gerekmekte, dosya ayarlarında değişiklik yapılmaması önerilmektedir.

% Belgenin formatı değiştirilmemelidir.
\documentclass[conference, a4paper]{IEEEtran}
\IEEEoverridecommandlockouts

% Türkçe karakterler için:
\usepackage[turkish]{babel}
\usepackage[utf8]{inputenc} % Kullanılan kodlamaya göre "utf8" yerine "latin5" de tercih edilebilir.
\usepackage[T1]{fontenc}

% ".jpeg" ve ".png" gibi görüntülerin belgede kullanılması için:
\usepackage[pdftex]{graphicx}

% Tablolarda birden fazla satır kullanılabilmesi için:
\usepackage{multirow}

% Referansların verilmesi ve numaralandırılması için:
\usepackage{cite}

% Matematiksel gösterimler için:
\usepackage[cmex10]{amsmath}

% SI ölçü birimlerinin kullanılabilmesi için:
\usepackage{siunitx}

% Matrisler ve diğer dizilerin kullanılabilmesi için:
\usepackage{array}

% Alt şekillerin değiştirilip, referans edilebilmesi için:
\usepackage[caption=false,lofdepth,lotdepth]{subfig}

% Kısaltmalar için:
\usepackage{acronym}
% Örnek Kısaltma
\acrodef{TUAC}{The Undergraduate Academic Conference}

% Hatalı hecelemeler örnekteki gibi düzeltilebilir
\hyphenation{op-tical net-works semi-conduc-tor}

\setlength{\textfloatsep}{5pt}

\AtBeginDocument{%
	\renewcommand\tablename{TABLO}
}

\AtBeginDocument{%
	\renewcommand\abstractname{Abstract}
}

\begin{document}
% Bildiri başlığı
% Daha iyi bir biçimlendirme için \linebreak komutu satır atlamak için kullanılabilir
\title{Bildiri Başlığı (Türkçe)\\
	Title of Paper (In English)}

% Yazar isimleri ve bağlantıları
% Üç farklı yazar için çoklu sütun kullanılmalıdır
\author{\IEEEauthorblockN{1. Ad Soyad}
	\IEEEauthorblockA{\textit{Bölüm}\\
		\textit{Üniversite}\\
		Şehir, Ülke \\
		e-posta adresi}
	\and
	\IEEEauthorblockN{2. Ad Soyad}
	\IEEEauthorblockA{\textit{Bölüm} \\
		\textit{Üniversite}\\
		Şehir, Ülke \\
		e-posta adresi}
	\and
	\IEEEauthorblockN{3. Ad Soyad}
	\IEEEauthorblockA{\textit{Bölüm} \\
		\textit{Üniversite}\\
		Şehir, Ülke \\
		e-posta adresi}
}

	% Başlığın yazdırılması
	\maketitle
	
	\begin{ozet}
		Bu belge, TUAC bildirisi hazırlanması adına bir taslak içermektedir. Bu sebeple lütfen taslaktaki başlık, özet ve diğer format stillerini kullanınız.  \textit{*DİKKAT:  Bildiri Başlığı'nda ve Özetçe'de Sembol, Özel ve Matematiksel Karakterler kullanmayınız.}
		%\boldmath
	\end{ozet}
	\begin{IEEEanahtar}
		doküman biçimi, stil, anahtar kelimeler.
	\end{IEEEanahtar}
	
	\begin{abstract}
		This electronic document is a template for TUAC and already defines the components of your paper [title, text, heads, etc.] in its style sheet.  \textit{*CRITICAL:  Do Not Use Symbols, Special Characters, or Math in Paper Title or Abstract.}
	\end{abstract}
	\begin{IEEEkeywords}
		component, formatting, style, styling, insert (key words).
	\end{IEEEkeywords}
	
	\IEEEpeerreviewmaketitle
	
	\IEEEpubidadjcol
	
	
	\section{G{\footnotesize İ}r{\footnotesize İ}ş}
	
	Bu taslak, IEEE'nin konferans bildirisi yazımı için önerdiği {\LaTeX} taslağı kullanılarak  hazırlanmıştır. Kenar boşlukları, sütun genişlikleri, satır aralıkları ve stiller, taslağın içine gömülü olarak gelmektedir.  Taslağa bağlı kalmaya dikkat ediniz.
	
	\section{Kullanım}
	
	\subsection{Taslak Seçmek}
	
	Doğru taslağı (bu taslağı) kullandığınızdan emin olunuz.
	
	\subsection{Taslağın Formatına Bağlı Kalmak}
	
	Taslağın formatını değiştirmeyiniz. Taslak üzerinde oynamalar yapmak, derleme hatalarına ve değerlendirme sürecinde zorluklar yaşanmasına sebep olacaktır.
	

	\section{Sayfa Düzen{\footnotesize İ} ve B{\footnotesize İ}ç{\footnotesize İ}m}
	
	Belgeyi düzenlemeye başlamadan önce tüm çalışmanızı ayrı bir dosya olarak kaydetmeniz tavsiye edilir. Ayrıca düzenleme sonuçlanıncaya kadar grafik ve şekilleri düz yazıdan ayrı tutmanız faydalı olacaktır. Çalışmanın herhangi bir noktasında sayfa numaralandırması yapılmamalıdır. Taslak içerisindeki başlıklar numaralandırılacağından, ayrıca sizin numaralandırmanıza gerek yoktur. Sayfa düzenlenirken aşağıdaki kurallara uyulmalıdır. Hazır bir taslak (Microsoft Word ya da {\LaTeX}) kullanmanız veya ayrıntıların kontrolü için örnek bir dosya takip etmeniz, bu gereklilikleri yerine getirmeniz açısından önerilir.
	
	\subsection{Kısaltmalar}
	
	Kısaltmaları, yazı içinde ilk defa kullanıldıklarında tanımlayınız. Başlıklarda kısaltma kullanmayınız. IEEE, SI, CGS vb. gibi çok bilinmiş kısaltmaları tanımlamanıza gerek yoktur.
	
	\subsection{Birimler}
	
	\begin{itemize}
		\item SI veya CGS ölçü birimlerini kullanınız. (SI ölçü birimi tavsiye edilir.)
		\item Yazı içinde farklı ölçü birimleri kullanmayınız. İngiliz ölçü birimlerini birinci birim olarak kullanmaktan kaçınınız. İngiliz ölçü birimlerini gerektiren bir durum söz konusu ise parantez içerisinde ikinci birim olarak gösteriniz.
		\item Ölçü birimlerini yazarken tutarlılık sağlayınız. Örneğin ``\si{\weber\per\square\meter}'' veya ``webers per square meter'' kullanınız, ``webers/\si{\square\meter}'' kullanmayınız.
		\item Küsuratlı sayı kullanırken ``.25'' yerine ``0.25'' kullanınız.
	\end{itemize}
	
	\subsection{Denklemler}
	
	Denklemler taslaktaki formata özeldir ve aşağıdaki örneğe benzemelidir.
	
	\begin{equation}
		\lim_{x \to \infty} \exp(-x) = 0
		\label{eq_1}
	\end{equation}
	
	Denklemler ortaya hizalanmış olmalıdır. Denklemdeki sembolleri tanımladığınızdan emin olunuz. Denklemlerden ``\eqref{eq_1}'' şeklinde bahsediniz. Cümle başında ``Denklem \eqref{eq_1}'' olarak kullanabilirsiniz.
	
	
	\subsection{Kaynak Formatı}
	
	Bildiride kullanılan kaynaklar ``Kaynaklar'' bölümünde IEEE kaynak gösterim formatı kullanılarak listelenmelidir. Örnek bir ``Kaynaklar'' bölümü bu taslağın sonunda gösterilmiştir.
	
	\section{Taslağı Kullanmak}
	
	\subsection{Yazarlar}
	
	Bildirinin yazarlarına ait isimler başlığın altında ve taslakta konumlandırıldığı şekilde belirtilmelidir. Yazarların bölüm, üniversite, şehir, ülke bilgileri ve iletişim e-posta adresleri alt satırlarda belirtildiği şekilde verilmelidir.
	
	\subsection{Başlıklar}
	
	Bölüm başlıklarını bu taslakta kullanılan temel {\LaTeX} komutlarını kullanarak düzenleyiniz. Bölüm isimlerinde ``küçük-büyükharf (small caps)'' fontu kullanıldığından, bölüm isimlerini ``\verb=\section{KULLANIM}='' şeklinde değil ``\verb=\section{Kullanım}='' şeklinde yazınız. Eğer isimde ``i'' harfi mevcut ise ``\verb=\section{G{\footnotesize İ}r= \verb={\footnotesize İ}ş}='' şeklinde yazınız.
	
	Eğer birden fazla alt konu yoksa, alt konu başlığı kullanmayınız.
	
	\subsection{Şekil ve Tablolar}
	
	\subsubsection{Şekil ve tabloların yerleştirilmeleri} Şekilleri ve tabloları sütun başına veya sonuna yerleştiriniz. Şekil başlığını şeklin altına yerleştiriniz. Tablo başlığını tablonun üstüne yerleştiriniz. Tablo örneği ve şekil başlığı örneği aşağıda belirtilmiştir. Mümkün olduğunca tablo ve şekilleri, isimleri metinde geçtikten sonra yerleştiriniz. Şekillerin eklenmesinde kullanılan ``\verb=\includegraphics[]{}='' komutundan önce ``\verb|\shorthandoff{=}|'' komutunu, sonra ``\verb|\shorthandon{=}|'' komutunu yazınız. Ayrıca tablo başlıkları, bölüm başlıkları gibi ``küçük-büyükharf (small caps)'' fontu ile yazılması gerektiğinden ``\verb=\caption{\textsc{Tablo Başlığı}}='' şeklinde yazınız. ``i'' harfinin bulunması durumunda bu harfi ``\verb={\footnotesize İ}='' şeklinde gösteriniz.
	
	\subsubsection{Aksis tanımlamaları} 8 büyüklüğünde punto kullanınız. Kısaltma kullanmayınız. Birim ekleyecekseniz ``Sıcaklık/K'' değil, ``Sıcaklık (\si{\kelvin})'' şeklinde olmalıdır.
	
	\begin{table}[h]
		\centering
		\caption{\textsc{Örnek Tablo}}
		\label{tablo1}
		\begin{tabular}{|c|c|c|c|}
			\hline
			\multirow{2}{*}{Tablo Başlığı} & \multicolumn{3}{c|}{Tablo sütun başlığı} \\
			\cline{2-4}
			& Tablo sütun ara başlığı & Ara başlık & Ara başlık \\
			\hline
			& & & \\
			\hline
		\end{tabular}
	\end{table}

	\begin{figure}[h]
		\centering
		\shorthandoff{=}  %\usepackage[turkish]{babel} kullanıldığı için komuta gerek var
		\includegraphics[scale=0.125]{resim.png}
		\shorthandon{=} %\usepackage[turkish]{babel} kullanıldığı için komuta gerek var
		\caption{Şekil örneği.}
		\label{sekil1}
	\end{figure}
	

	\section*{B{\footnotesize İ}lg{\footnotesize İ}lend{\footnotesize İ}rme}
	Bilgilendirmeler bu bölümde yapılabilir. Sponsor bilgilendirmeleri ilk sayfada dipnot olarak verilmelidir. İhtiyaç duyulmaması halinde bu bölüm kaldırılabilir.
	
	% Kaynaklarda IEEE Referans formatına uyulmalıdır.
	\begin{thebibliography}{1}
		\bibitem{paper1}
		M. Shand, S. Bryant, ``IP Fast Reroute Framework'', RFC 5714, 2010.
		
		\bibitem{paper2}
		A. Atlas, A. Zinin, ``Basic Specification for IP Fast Reroute: Loop-Free Alternates'', RFC 5286, 2008.
		
		\bibitem{paper3}
		S. Bryant, S. Previdi, M. Shand, ``IP Fast Reroute Using Not-via Addresses'', IETF Internet Draft, 2012.	
	\end{thebibliography}
\end{document} 